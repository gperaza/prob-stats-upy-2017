\documentclass{beamer}

\usetheme[sectionpage=simple, subsectionpage=simple,
numbering=fraction, progressbar=none]
{metropolis}

\usepackage[utf8]{inputenc}
\usepackage{graphicx}
\graphicspath{{./img/}}
\usepackage{tikz}

\title{Estimation}
\date{\today}
\author{Gonzalo G. Peraza Mues}
% \institute{Universidad Politécnica de Yucatán}

\titlegraphic{\includegraphics[height=1.5cm]{logo-upy}}

\newcommand{\E}[1]{\operatorname{E}\left[#1\right]}
\newcommand{\Var}[1]{\operatorname{Var}\left(#1\right)}
\newcommand{\abs}[1]{\left|#1\right|}
\renewcommand{\P}[1]{P\left(#1\right)}


\begin{document}
\maketitle
\begin{frame}
  A recent poll of 1500 randomly chosen people indicated that 22 percent of the
  entire population is presently dieting, with a margin of error of $\pm$2 percent.

  \begin{itemize}
  \item What does ``margin of error'' mean?
  \item How can the proportion of a population in the millions can be
    ascertained by sampling only 1500 people?
  \end{itemize}
\end{frame}

\begin{frame}{Estimators}
  When the distribution of the population is known but one or some of its
  parameters are missing one can use the results from a sample of the population
  to estimate these unknown parameters.

  \textbf{Definition: }An \alert{estimator} is a statistic whose value depends on
  the particular sample drawn. The value of the estimator, called the
  \alert{estimate}, is used to predict the value of a population parameter.
\end{frame}

\begin{frame}{Point estimators}

  \begin{itemize}
  \item The sample mean $\bar{X}$.
  \item The population proportion $\hat{p}$.
  \item The population variance $S^2$.
  \end{itemize}

  \textbf{Definition: }An estimator whose expected value is equal to the
  parameter it is estimating is said to be an \alert{unbiased} estimator of that
  parameter.

\end{frame}

\begin{frame}{Point Estimator of the Population Mean.}
  \begin{itemize}
  \item The estimator of the population mean $\mu$ is the sample mean $\bar{X}$.
  \item It is an unbiased estimator; $\E{\bar{X}}=\mu$.
  \item The standard error is $SEM=\sigma/\sqrt{n}$.
  \end{itemize}

  The estimate of the population mean will be correct to within $\pm$2 standard
  errors. To cut the standard error in half, we must increase the sample size by
  a factor of 4.
\end{frame}

\begin{frame}{Point Estimator of a Population Proportion}
  \begin{itemize}
  \item The estimator of the population proportion $p$ is the sample proportion
    $\hat{p}$.
  \item It is an unbiased estimator; $\E{\hat{p}}=p$.
  \item The standard error is $SD(\hat{p})=\sqrt{\frac{p(1-p)}{n}}$.
  \end{itemize}

  It can be shown that $p(1-p)\leq\frac{1}{4}$ which in turn means that
  $SD(\hat{p}) \leq \frac{1}{2\sqrt{n}}$.
\end{frame}

\begin{frame}{Estimating a Population Variance}
  \begin{itemize}
  \item The estimator of the population variance $\sigma^2$ is:
    \begin{itemize}
    \item $\frac{\sum_{i=1}^n (X_i-\mu)^2}{n}$if $\mu$ is known.
    \item $S^2$ if $\mu$ is unknown.
    \end{itemize}
  \item $S^2$ is an unbiased estimator; $\E{S^2}=\sigma^2$.
  \end{itemize}

  The population standard deviation $\sigma$ is estimated by $S$.
\end{frame}

\begin{frame}{Interval Estimators of the Mean with Known
    Population Variance}
  \textbf{Definition: }An \alert{interval estimator} of a population parameter
  is an interval that is predicted to contain the parameter. The
  \alert{confidence} we ascribe to the interval is the probability that it will
  contain the parameter.

  To determine an interval estimator of a population parameter, we use the prob-
  ability distribution of the point estimator of that parameter.
\end{frame}

\begin{frame}{Interval Estimators of the Mean with Known
    Population Variance}
  Let's obtain the $1-\alpha$ percent confidence interval estimator for $\mu$:
  \begin{align*}
    \P{\frac{\sqrt{n}}{\sigma}\abs{\bar{X}-\mu} \leq z_{\alpha/2}} = 1-\alpha
  \end{align*}

  Reorganizing
  \begin{align*}
    \P{\bar{X} - z_{\alpha/2}\frac{\sqrt{n}}{\sigma}
    \leq \mu \leq \bar{X} + z_{\alpha/2}\frac{\sqrt{n}}{\sigma}} = 1 - \alpha
  \end{align*}

  If the observed value of $\bar{X}$ is $\bar{x}$, then we call the interval
  \begin{align*}
    \bar{x}\pm z_{\alpha/2}\frac{\sigma}{\sqrt{n}}
  \end{align*}
  percent confidence interval estimate of $\mu$.
\end{frame}

\begin{frame}[t,shrink=10]{Interval Estimators of the Mean with Known
    Population Variance. Example}
  Suppose that if a signal having intensity $\mu$ originates at location A, then
  the intensity recorded at location B is normally distributed with mean $\mu$
  and standard deviation 3. To reduce the error, the same signal is
  independently recorded 10 times. If the successive recorded values are
  \begin{center}
    17, 21, 20, 18, 19, 22, 20, 21, 16, 19
  \end{center}
  construct a 90, 95 and 99 percent confidence interval for $\mu$, the actual
  intensity.
\end{frame}

\begin{frame}{Interval Estimators of the Mean with Known Population
    Variance. How Large a Sample is Needed?}
  The length of the $100(1 - \alpha)$ percent confidence interval estimator of
  the population mean will be less than or equal to $b$ when the sample size $n$
  satisfies
  \begin{align*}
    n \geq \left( \frac{2z_{\alpha/2}\sigma}{b}  \right)^2
  \end{align*}
\end{frame}

\begin{frame}[t,shrink=10]{Interval Estimators of the Mean with Known
    Population Variance. Example}
  From past experience it is known that the weights of salmon grown at a
  commercial hatchery are normal with a mean that varies from season to season
  but with a standard deviation that remains fixed at 0.3 pounds. If we want to
  be 90 percent certain that our estimate of the mean weight of a salmon is
  correct to within $\pm$0.1 pounds, how large a sample is needed? What if we
  want to be 99 percent certain?
\end{frame}

\begin{frame}{Interval Estimators of the Mean with Known
    Population Variance. Lower and Upper Confidence Bounds}
  A $100(1 - \alpha)$ percent lower confidence bound for the population mean
  $\mu$ is given by
  \begin{align*}
    \bar{X} - z_{\alpha}\frac{\sigma}{\sqrt{n}}
  \end{align*}

  A $100(1 - \alpha)$ percent upper confidence bound for the population mean
  $\mu$ is given by
  \begin{align*}
    \bar{X} + z_{\alpha}\frac{\sigma}{\sqrt{n}}
  \end{align*}
\end{frame}
\begin{frame}{Interval Estimators of the Mean with Unknown Population Variance}
  Since $\sigma$ is no longer known, we replace it by its estimator $S$. We
  based our confidence interval in the t-distributed variable:
  \begin{align*}
    T_{n-1} = \sqrt{n}\frac{\bar{X}-\mu}{S}
  \end{align*}

  A $100(1 - \alpha)$ percent confidence interval estimator for the population
  mean $\mu$ is given by the interval
  \begin{align*}
    \bar{X}\pm t_{n-1, \alpha/2}\frac{S}{\sqrt{n}}
  \end{align*}
\end{frame}

\begin{frame}{Interval Estimators of the Mean with Unknown Population Variance}
  A $100(1 - \alpha)$ percent lower confidence bound for the population mean
  $\mu$ is given by
  \begin{align*}
    \bar{X} - t_{n-1,\alpha}\frac{S}{\sqrt{n}}
  \end{align*}

  A $100(1 - \alpha)$ percent upper confidence bound for the population mean
  $\mu$ is given by
  \begin{align*}
    \bar{X} + t_{n-1,\alpha}\frac{S}{\sqrt{n}}
  \end{align*}
\end{frame}

\begin{frame}[t,shrink=10]{Interval Estimators of the Mean with Unknown Population
    Variance: Example}
  The Environmental Protection Agency (EPA) is concerned about the amounts of
  PCB, a toxic chemical, in the milk of nursing mothers. In a sample of 20
  women, the amounts (in parts per million) of PCB were as follows:
  \begin{center}
    16, 0, 0, 2, 3, 6, 8, 2, 5, 0, 12, 10, 5, 7, 2, 3, 8, 17, 9, 1
  \end{center}
  Use these data to obtain a (a) 95 percent confidence interval; (b) 99 percent
  confidence interval of the average amount of PCB in the milk of nursing
  mothers. (c) 95 percent upper confidence bound (d); 99 percent lower
  confidence bound
\end{frame}


\begin{frame}{Interval Estimators of a Population Proportion}
  We can use the normal approximation to the binomial distribution to obtain the
  approximate $100(1-\alpha)$ percent confidence interval estimator of $p$
  \begin{align*}
    \hat{p} \pm z_{\alpha/2}\sqrt{\frac{\hat{p}(1-\hat{p})}{n}}
  \end{align*}

  The length of this interval is
  \begin{align*}
    2z_{\alpha/2}\sqrt{\frac{\hat{p}(1-\hat{p})}{n}}
  \end{align*}
  with the upper bound
  \begin{align*}
    \text{Length of interval} \leq \frac{z_{\alpha/2}}{\sqrt{n}}
  \end{align*}
\end{frame}

\begin{frame}{Interval Estimators of a Population Proportion}
  A $100(1 - \alpha)$ percent lower confidence bound for the population mean
  $\mu$ is given by
  \begin{align*}
    \hat{p} - z_{\alpha}\sqrt{\frac{\hat{p}(1-\hat{p})}{n}}
  \end{align*}

  A $100(1 - \alpha)$ percent upper confidence bound for the population mean
  $\mu$ is given by
  \begin{align*}
    \hat{p} + z_{\alpha}\sqrt{\frac{\hat{p}(1-\hat{p})}{n}}
  \end{align*}
\end{frame}

\begin{frame}[t]{Interval Estimators of a Population Proportion. Example}
  Out of a random sample of 100 students at a university, 82 stated that they
  were nonsmokers. Based on this, construct a 99 percent confidence interval
  estimate of p, the proportion of all the students at the university who are
  nonsmokers.
\end{frame}

\begin{frame}[t]{Interval Estimators of a Population Proportion. Example}
  How large a sample is needed to ensure that the length of the 90 percent
  confidence interval estimate of p is less than 0.01?
\end{frame}
\end{document}

%%% Local Variables:
%%% mode: latex
%%% TeX-master: t
%%% End:
